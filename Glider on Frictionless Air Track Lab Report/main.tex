\documentclass[12pt]{article}
    
\usepackage[margin=1.25in]{geometry} 
\usepackage{amsmath,amsthm,amssymb} 
\usepackage{tikz}
\usepackage{indentfirst}
\usepackage{amsmath}

%\linespread{1.5}

\begin{document}

\title{Balcony Height Lab Report}
\author{AP Physics | B Block\\
\\
Ryan McCrystal\\
Kala'i Anderson\\
Bella Otterson}
\date{\today}
\maketitle
\newpage
\begin{abstract} % Include purpose, methodologies, assumptions, and brief statement of results
The purpose of this experiment is to calculate the height of the Cooper House balcony as accurately as possible and to determine which method is the most accurate. Two different methods were used to calculate the height of the balcony. The first method we used involved creating two similar triangles by placing a pole of known length a distance away from the ground directly below the balcony. Using this setup, one of our group members aligned their eye to be col-linear to the top of the pole and the balcony. Using these measurements, we were able to mathematically construct two similar triangles giving us a calculated height of $15.4ft\pm1.4ft$. The second method we used to indirectly calculate the height of the balcony involved a tennis ball, a stopwatch, and the appropriate kinematic equations. One of our group members dropped a ball at the top of the railing while another group member timed the amount of time it took for the ball to hit the ground. Because we know the ball's acceleration due to the constant force of gravity, we could indirectly calculate the height of the railing. This method provided us with a much more inaccurate result of  $12.6ft\pm8.12ft$. 
\end{abstract}
\newpage
\end{document}